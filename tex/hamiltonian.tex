\NeedsTeXFormat{LaTeX2e}
\documentclass[11pt]{article}
%Absaetze nicht einruecken:
\usepackage{parskip}
\usepackage[utf8]{inputenc}
\usepackage[T1]{fontenc}
\usepackage{ae}
\usepackage[intlimits, sumlimits, namelimits]{amsmath}
\usepackage{bbm}
%Neue Macros fuer Mathe:
%in parentheses - gleich mit richtiger Groesse
\newcommand{\inp}[1]{\ensuremath{\left(#1\right)}}
\newcommand{\sqr}{\ensuremath{^{2}}}
\newcommand{\cube}{\ensuremath{^{3}}}
%Mengensymbole mit doppelten senkrechten Strichen:
\newcommand{\set}[1]{\ensuremath{\mathbbm{#1}}}
%definiert eine Norm, also zwei senkrechte Striche auf jeder Seite:
\newcommand{\norm}[1]{\ensuremath{\left|#1\right|}}
%Spaltenvektor - dreidimensional:
\newcommand{\svec}[3]{\ensuremath{\inp{\hspace{-.8ex}\begin{array}{r}#1\\#2\\#3\end{array}\hspace{-.4ex}}}}
\newcommand{\entspr}{\ensuremath{\,\,\hat{=}\,\,}}%
\newcommand{\dx}[1][x]{\ensuremath{\textnormal d #1}}

% For stupid thinkos:
\newcommand{\cross}{\times}

% Tabellen:
\usepackage{array}
\setlength{\extrarowheight}{.2mm}
%Links im Text:
\usepackage{hyperref}
\usepackage{graphics,graphicx,fancyvrb}
%Raender einstellen
%\usepackage[a4paper, margin=15mm, top=30mm]{geometry}
\usepackage[a4paper]{geometry}
%Kopf - und Fusszeilen
\usepackage{lastpage}
\usepackage{fancyhdr}
	\lhead{Moritz Lenz}
	\chead{\bfseries{-- \thepage\ --}}
	\rhead{\thetitle}
	\lfoot{}
	\rfoot{}
	\cfoot{}
	\pagestyle{fancy}
\pagestyle{empty}

\sffamily

%Kopfzeile

\author{Moritz Lenz}
\title{Tight binding Hamiltonian}
\begin{document}
\maketitle

In order to numerically simulate electrical and spin transport in the 2DEG, we
discretize the sample into sites and assume that electrons only travel from
one site to an immediate neighbor.

For a two-dimensional electron gas with Rashba spin-orbit coupling (of
strength $\alpha$) the Hamiltonian is

\begin{equation}
    H = \frac{1}{2 m^*} (p_x^2 + p_y^2) + 
    \frac{\alpha}{\hbar} \inp{p_y\sigma_x - p_x\sigma_y}
\end{equation}

Taking in account only nearest neighbor interactions, and discretizing to a
tight binding model we get

\begin{eqnarray}
    H   &&= H_0 + H_r\\
    H_0 &&= \sum_{n,\sigma} \epsilon_0 c^{\dagger}_{n\sigma} c_{n\sigma}
            - \sum_{n,\delta,\sigma} t c^\dagger_{n\sigma} c_{n,\sigma} +
            \textnormal{H.c.}\\
    H_r &&= \frac{-\alpha}{2 a_0} \sum_m
        -i( c^\dagger_{m,\uparrow|} c_{m+a_y,\downarrow}
            + c^\dagger_{m,\downarrow} c_{m+a_y,\uparrow})
         + c^\dagger_{m,\uparrow|} c_{m+a_x,\downarrow}
            + c^\dagger_{m,\downarrow} c_{m+a_x,\uparrow}
\end{eqnarray}

where $n$ runs over all lattice sites, $\delta$ over $\uparrow$ and
$\downarrow$, and $a_x$ and $a_y$ denote the shift to the nearest neighbor in
$x$ and $y$ direction, respectively. (Note that we assume that the lattice is
equally spaced in $x$ and $y$ direction, $+a_x$ just means "go to the next
neighbor in $x$ direction).

Assume we have a quadratic system of $N \times N$ lattice sites.
For a two-dimensional system we enumerate all lattice sites row by row, and
use the result as the index to the Hamiltonian. To incorporate spin, we
identify the indexes that were assigned so far with spin-up, and add $N^2$ to
each index to obtain the matrix index for spin-down.

For example if our system were of size $3 \times 3$, the left-most site in the
first row has index $i = 1$, and the left-most site in the second row has
index $j = N + 1 = 4$ (both spin down). The interaction term (without spin
flip) between these two sites can thus be found at $H_{i,j} = H_{1,4}$. The
interaction that involves a spin flip from $\uparrow$ to $\downarrow$ is
described by $H_{i, j+N^2} = H_{1, 13}$.

\begin{figure}
    \begin{align*}
        H &&= \inp{
           \begin{array}{cc}
                H_{kin}  & H_{spin} \\
                H_{spin}^\dagger & H_{kin} \\
           \end{array}} \\
           %
        H_{kin} &&= \inp{
            \begin{array}{ccccccccc}
                -4t & t &  & t\\
                t & -4t & t &  & t &  &  & 0\\
                & t & -4t & 0 &  & t\\
                t &  & 0 & -4t & t &  & t\\
                & t &  & t & -4t & t &  & t\\
                &  & t &  & t & -4t & 0 &  & t\\
                &  &  & t &  & 0 & -4t & t\\
                & 0 &  &  & t &  & t & -4t & t\\
                &  &  &  &  & t &  & t & -4t\end{array}
        } \\
        %
        H_{spin} &&= \inp{
            \begin{array}{ccccccccc}
                0 & -r &  & r\\
                r & 0 & -r &  & r &  &  & 0\\
                & r & 0 & 0 &  & r\\
                -r &  & 0 & 0 & -r &  & r\\
                & -r &  & r & 0 & -r &  & r\\
                &  & -r &  & r & 0 & 0 &  & r\\
                &  &  & -r &  & 0 & 0 & -r\\
                & 0 &  &  & -r &  & r & 0 & -r\\
                &  &  &  &  & -r &  & r & 0\end{array}
        } 
    \end{align*}
    \caption{Tight binding Hamiltonian for $ 3 \times 3 $ lattice sites}
    \label{fig:hamiltonian}
\end{figure}

Figure \ref{fig:hamiltonian} shows an example Hamiltonian for a system of
$3 \times 3$ lattice sites with Rashba spin-orbit coupling (and no magnetic
field).

\end{document}

% vim: ts=4 sw=4 expandtab spell spelllang=en_us tw=78

\chapter{Summary and Outlook}
\label{sec:summary}

We analyzed the possibilities of achieving spin polarization in a
non-magnetic, microscopic semiconductor. We found that an interface between
normal and Rashba spin-orbit coupling areas splits an electron beam into
components with different chirality, and can be used to manipulate
spin-polarized electron beams similarly to optical light in media with
different optical densities.

           
If the angle between the interface
and the incident beam exceeds a critical angle, the beam with $+$ chirality
ceases to propagate. This effect can be used to obtain a spin imbalance.

We modeled such a system analytically in order to get a good understanding
of the physics, and in an extensible numerical simulation in order to have
maximal flexibility with the choice of parameters and system details.

We developed a method to compare the analytical and numerical results, and in
that course we found that the measurable spin imbalance is partially
damped by fact that the interface separates electron waves by chirality,
not by spin component in $z$-direction.

Still we predict a decent amount of spin-polarization in $z$ direction, which
is directly accessible to experimental verification.

We also discussed the experimental more accessible setup of two regions with
different, non-zero strengths of spin-orbit interaction, and found that such an
interface can also be used to achieve some spin polarization, albeit
of decreasing magnitude when the spin-orbit coupling strengths become similar. 

In both cases a large angle between the incident beam and the interface is
essential for obtaining a decent spin polarization.

We also pointed out that experimental realization is quite possible, and that
similar (but slightly more complex) setups have been grown and etched in HgTe
quantum wells.

Future work in this area could cover the experimental setup of having a strip
of tunable spin-orbit interaction.

There is also another simple extension to our model that would help making
quantitative predictions: when the spin-orbit coupling in an area is tuned by
applying a gate voltage, not only the spin-orbit coupling strength changes,
but also a potential barrier is built up, which scatters electron in a
spin-independent way. Taking these barriers into account will probably not
enhance our understanding of the spin polarization, but is crucial for
obtaining quantitative predictions of the measured signals.

Also in the experiments  magnetic fields are used to focus beams in the
sample, so magnetic field would be a worthwhile addition to our model.

To incorporate more realistic parameters of a particular semiconductor, 
our model could be expanded to use four bands,
the conductance and valance band for each spin direction. This would allow to
make quantitative recommendations on best angle for such an interface.


% vim: ts=4 sw=4 expandtab spell spelllang=en_us tw=78

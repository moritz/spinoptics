\chapter{Summary and Outlook}

We analyzed the possibilities of achieving spin polarization in a
non-magnetic, microscopic semiconductor. We found that an interface between
normal and Rashba spin-orbit coupling areas exhibits critical phenomena, which
can be used to achieve such a spin polarization.

We found that the measurable spin polarization is diminished by fact that the
interface separates electron waves by chirality, not by spin component
in $z$-direction.

We also discussed the experimental more accessible setup of two regions with
different, non-zero strength of spin-orbit interaction, and found that such an
interface can also be used to achieve some spin polarization, albeit
of decreasing magnitude when the spin-orbit coupling strengths become similar. 

In both cases a large angle between the incident beam the interface is
essential for a decent spin polarization.


Future work in this area could involve a four-band model which includes both
the conductance and valance band for each spin direction, would
allow more precises modeling of a particular semiconductor, and thus be of
more help to experimentalists. Using more realistic and specific parameters
would allow to make quantitative recommendations on which angle best to use
for such an interface.

%There is also another simple extension to our model that would help making
%quantitative predictions: when the spin-orbit coupling in an area is tuned by
%applying a gate voltage, not only the spin-orbit coupling strength changes,
%but also the electron density in the two-dimensional electron gas. So it also
%the effect of introducing a wall which 

% vim: ts=4 sw=4 expandtab spell spelllang=en_us tw=78

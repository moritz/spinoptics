\NeedsTeXFormat{LaTeX2e}
\documentclass[11pt]{article}
%Absaetze nicht einruecken:
\usepackage{parskip}
\usepackage[utf8]{inputenc}
\usepackage[T1]{fontenc}
\usepackage{ae}
\usepackage[intlimits, sumlimits, namelimits]{amsmath}
\usepackage{bbm}
%Neue Macros fuer Mathe:
%in parentheses - gleich mit richtiger Groesse
\newcommand{\inp}[1]{\ensuremath{\left(#1\right)}}
\newcommand{\sqr}{\ensuremath{^{2}}}
\newcommand{\cube}{\ensuremath{^{3}}}
%Mengensymbole mit doppelten senkrechten Strichen:
\newcommand{\set}[1]{\ensuremath{\mathbbm{#1}}}
%definiert eine Norm, also zwei senkrechte Striche auf jeder Seite:
\newcommand{\norm}[1]{\ensuremath{\left|#1\right|}}
%Spaltenvektor - dreidimensional:
\newcommand{\svec}[3]{\ensuremath{\inp{\hspace{-.8ex}\begin{array}{r}#1\\#2\\#3\end{array}\hspace{-.4ex}}}}
\newcommand{\entspr}{\ensuremath{\,\,\hat{=}\,\,}}%
\newcommand{\dx}[1][x]{\ensuremath{\textnormal d #1}}

% For stupid thinkos:
\newcommand{\cross}{\times}

% Tabellen:
\usepackage{array}
\setlength{\extrarowheight}{.2mm}
%Links im Text:
\usepackage{hyperref}
\usepackage{graphics,graphicx,fancyvrb}
%Raender einstellen
%\usepackage[a4paper, margin=15mm, top=30mm]{geometry}
\usepackage[a4paper]{geometry}
%Kopf - und Fusszeilen
\usepackage{lastpage}
\usepackage{fancyhdr}
	\lhead{Moritz Lenz}
	\chead{\bfseries{-- \thepage\ --}}
	\rhead{\thetitle}
	\lfoot{}
	\rfoot{}
	\cfoot{}
	\pagestyle{fancy}
\pagestyle{empty}

\sffamily

%Kopfzeile

\author{Moritz Lenz}
\title{Programming QA}
\begin{document}
\maketitle

Programming is constantly prone to subtle errors, so measures had to 
be taken to ensure that no errors slipped in that might lead to wrong
output.

The very first sanity check is that the Hamiltonian $H$ is indeed a hermitian
operator. This check of course only catches stupid programming
mistakes.

A more elaborate check is that $T_{pq}$ obeys the same symmetries as
the Hamiltonian. Since it can be written as

\begin{equation}
H = \frac{\vec{p}^2}{2 m^*} + \vec \sigma \cdot (\vec e_z \times \vec p)  
    + \vec \sigma \cdot \vec B
\end{equation}

It is easy to see that $H(\vec p, \vec \sigma, \vec B) = H(-\vec p,
-\vec \sigma, -\vec B)$, so the transmission matrix must follow the
same symmetries.

Another condition to check is derived by Datta (page TODO):

\begin{equation}
    \sum_p T_{pq} = \sum_p T_{qp} = M \qquad \forall q
\end{equation}

\end{document}

% vim: ts=4 sw=4 expandtab spell spelllang=en_us tw=70

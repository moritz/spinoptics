\chapter{Introduction}

Spin manipulation opens up an interesting range of possible applications, from
spin powered nano devices to Quantum Computing and Quantum Cryptography.

A quantum computer works with superposition of quantum mechanical states, and
spin states stay coherent for much longer than change states, so spin states
present a natural alternative to the classical systems which use spatially
separated electron charges to store and manipulate information.

Also in charge based devices -- such as NPN transistors -- charges have to be
moved to change the conduction properties, and moving charge requires energy,
and dissipates heat. Since spin devices are conceivable which don't require
spatial separation to operate, but rather work with flipping spins, they could
be much more energy efficient.

The first measurement of the spin degree of freedom was carried out by O.~Stern
and W.~Gerlach in 1921\cite{stern-gerlach}, and employed a beam of silver atoms,
which were
exposed to a strong, inhomogeneous magnetic field and thus spatially separated
according to their spin orientation.

For building spin based devices, such a setup is hardly feasible. Later
experiments in solids used ferromagnetic materials generate spin polarized
electron beams. A famous example is the Giant Magneto Resistance, which
enabled much higher storage density in hard disks, and was awarded with the
Nobel Prize in 2007.

However spin manipulation has traditionally been hard to achieve on nanometer
scales. Using ferromagnetic materials it is possible inject spin polarized
currents into semiconductor structures, and various mechanisms have been
researched to manipulate these spin currents.
However building ferromagnetic devices on
the nano scale is a serious technological challenge, and combining millions of
ferromagnetic structures on a single seems hardly possible.

In this Diploma Thesis we therefore investigate how a spin polarized electron
beam can be achieved by using only non-magnetic materials. The Rashba 
spin-orbit coupling, which arises from
asymmetric structures in certain semiconductors, can be used as a tunable
means to treat
electrons in a spin-dependent way. In particular an interface
between regions with different strengths of spin orbit interactions can be
used to split a non-polarized beam into two spatially separated beams of
different spin polarization.

We discuss such interfaces, and also the experimentally more accessible setup
of having two regions with different strengths of spin-orbit coupling.

In particular we look at a nanometer or micrometer sized, two dimensional
electron gas in a semiconductor at zero temperature, where electrons and holes
are transported coherently (ie without dephasing) and ballistically (ie
without or with very little scattering). Such electron gases are
experimentally accessible in quantum wells at heterojunctions in GaAs,
HgTe and other semiconductors.

We present both analytical calculations and numerical simulations based on
tight-binding approximation. The analytical calculations show an analogy to
light optics:  an interface between two media with different optical densities
can result in polarized light, and polarization dependent reflection. In
analogy an interface between two regions of different spin-orbit coupling
strength splits up an electron beam into two beams of defined chirality.

% vim: spell

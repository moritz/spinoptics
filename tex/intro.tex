\chapter{Introduction}

Spin manipulation opens up an interesting range of possible applications, from
spin powered nano devices to Quantum Computing and Quantum Cryptography.

A quantum computer works with superposition of quantum mechanical states, and
spin states stay coherent for much longer than change states, so spin states
present a natural alternative to the classical systems which use spatially
separated electron charges to store and manipulate information.

Also in charge based devices -- such as NPN transistors -- charges have to be
moved to change the conduction properties, and moving charge requires energy,
and dissipates heat. Since spin devices are conceivable which do not require
spatial separation to operate, but rather work with flipping spins, they could
be much more energy efficient.

The first measurement of the spin degree of freedom was carried out by O.~Stern
and W.~Gerlach in 1921 \cite{stern-gerlach} and employed a beam of silver atoms,
which were
exposed to a strong, inhomogeneous magnetic field and thus spatially separated
according to their spin orientation.

For building spin based devices, such a setup is hardly feasible. Later
experiments in solids used ferromagnetic materials generate spin polarized
electron beams. A famous example is the Giant Magneto Resistance, which
enabled much higher storage density in hard disks, and was awarded with the
Nobel Prize in 2007.

Spin manipulation has traditionally been hard to achieve on nanometer
scales. Using ferromagnetic materials, it is possible inject spin polarized
currents into semiconductor structures, and various mechanisms have been
researched to manipulate these spin currents. In 1990 S.~Datta and B.~Das
published their thoughts on how to build a spin-based
transistor\cite{datta-das}.

They proposed two ferromagnetic contacts separated
by quantum well with tunable spin-orbit coupling. By tuning the spin-orbit
coupling strength, the spin precession length can be controlled, and thus the
spin orientation at the second contact. Since ferromagnetic contacts are
sensitive to spin orientation, the current through the contacts can be
modulated.

However, building ferromagnetic contacts or devices on
the nano scale is a serious technological challenge, and combining millions of
ferromagnetic structures on a single device seems hardly possible. There is
also a conceptual difficulty: due to the band structure mismatch between
metallic ferromagnetic materials and semiconductors, additional interface
effects (like in Schottky diodes) arise, which can seriously inhibit the
usefulness of such devices.

New hope for non-magnetic spintronic devices came from the experimental
observation of the Spin-Hall Effect in 2004 \cite{SHE}. In analogy to the
classical Hall effect, an electrical current causes a spin imbalance in
lateral direction. Unlike the ordinary Hall effect, no magnetic field is
required, but rather the spin assembly is caused by the band
structure of the semiconductor hetereostructure, or by impurities.

In this Diploma Thesis, we investigate how a spin polarized electron
beam can be achieved by using only non-magnetic materials. The Rashba 
spin-orbit coupling, which arises from
asymmetric structures in certain semiconductors, can be used as a tunable
means to treat
electrons in a spin-dependent way. In particular, an interface
between regions with different strengths of spin orbit interactions can be
used to split a non-polarized beam into two spatially separated beams of
different spin polarization.

We discuss such interfaces, and also the experimentally more accessible setup
of having two regions with different strengths of spin-orbit coupling.

In particular we look at a nanometer or micrometer sized, two dimensional
electron gas in a semiconductor at zero temperature where electrons and holes
are transported coherently (i.e.~without dephasing) and ballistically
(i.e.~without or with very little scattering). Such electron gases are
experimentally accessible in quantum wells at heterojunctions in GaAs,
HgTe and other semiconductors.

In Chapter \ref{sec:theory} we present the basic theoretical underpinning for
the calculations to come: the Landauer Formula which relates conductance to
the transmission matrix $T$, the Fisher-Lee relation which allows calculation
of $T$ based on the Green's function in sample and lead, the Rashba-Bychkov
spin-orbit coupling which causes all the interesting effects discussed in this
thesis, and finally we present a tight binding model which allows numerical
calculation of the Green's functions and thus $T$.

In Chapter \ref{sec:analytical} we present an analytical model of an electron
wave traveling from a normal region to a region with spin-orbit coupling. The
wave is decomposed into two parts of opposite chirality, and we 
analyze the transmission and reflection coefficients resolved by chirality.
We expand this model to a system where both sides of the interface have
non-zero (but different) spin-orbit interaction.

Chapter \ref{sec:numerics} explains the numerical calculations in depth. We
present the used algorithm and possible alternatives, considerations regarding
the run time performance and numerical errors, and of course results from
these calculations. We find that decent spin polarizations can be achieved
for appropriate interface angles and spin-orbit strengths, even in the new
case where there is non-zero spin-orbit interaction on both sides of
the interface.

We also explain how the results of the analytical calculations can be compared
to the numerical results, and how projection from the chiral bases to the
spin-up/spin-down bases diminishes some of effects of the interfaces.

Chapter \ref{sec:summary} finally summarizes our achievements, and shows up
possible directions in which our models could be expanded.

% vim: spell

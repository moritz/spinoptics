Spin manipulation opens up an interesting range of possible applications, from
spin powered nano devices to Quantum Information and Quantum Cryptography.

However spin manipulation has traditionally been hard to achieve on nanometer
scales. Using ferromagnetic materials it is possible inject spin polarized
currents into semiconductor structures, but they are rather hard to handle
and build.

In this Diploma Thesis we therefore investigate spin manipulation in
non-magnetic material. The Rashba spin-orbit coupling, which arises from
asymmetric structures in certain semiconductors, can be used to treat
electrons in a spin-dependent way. In particular the physics of interfaces
between regions with different strengths of spin orbit interactions are
discussed, and how they might be used to generate spin polarized electron
beams.

In particular we look at a nanometer or micrometer sized, two dimensional
electron gas in a semiconductor at zero temperature, where electrons and holes
are transported coherently (ie without dephasing) and ballistically (ie
without or with very little scattering). Such an electron gas is
experimentally accessible at heterojunctions in GaAs, HgTe and other semiconductors.

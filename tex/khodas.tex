\NeedsTeXFormat{LaTeX2e}
\documentclass[11pt]{article}
%Absaetze nicht einruecken:
\usepackage{parskip}
\usepackage[utf8]{inputenc}
\usepackage[T1]{fontenc}
\usepackage{ae}
\usepackage[intlimits, sumlimits, namelimits]{amsmath}
\usepackage{bbm}
%Neue Macros fuer Mathe:
%in parentheses - gleich mit richtiger Groesse
\newcommand{\inp}[1]{\ensuremath{\left(#1\right)}}
\newcommand{\sqr}{\ensuremath{^{2}}}
\newcommand{\cube}{\ensuremath{^{3}}}
%Mengensymbole mit doppelten senkrechten Strichen:
\newcommand{\set}[1]{\ensuremath{\mathbbm{#1}}}
%definiert eine Norm, also zwei senkrechte Striche auf jeder Seite:
\newcommand{\norm}[1]{\ensuremath{\left|#1\right|}}
\newcommand{\vect}[2]{\ensuremath{\inp{\hspace{-.8ex}\begin{array}{c}#1\\#2\end{array}\hspace{-.4ex}}}}
%\newcommand{\vec3}[3]{\ensuremath{\inp{\hspace{-.8ex}\begin{array}{r}#1\\#2\\#3\end{array}\hspace{-.4ex}}}}
\newcommand{\entspr}{\ensuremath{\,\,\hat{=}\,\,}}%
\newcommand{\dx}[1][x]{\ensuremath{\textnormal d #1}}

\newcommand{\ta}{\tilde \alpha}

% For stupid thinkos:
\newcommand{\cross}{\times}

% Tabellen:
\usepackage{array}
\setlength{\extrarowheight}{.2mm}
%Links im Text:
\usepackage{hyperref}
\usepackage{graphics,graphicx,fancyvrb}
%Raender einstellen
%\usepackage[a4paper, margin=15mm, top=30mm]{geometry}
\usepackage[a4paper]{geometry}
%Kopf - und Fusszeilen
\usepackage{lastpage}
\usepackage{fancyhdr}
	\lhead{Moritz Lenz}
	\chead{\bfseries{-- \thepage\ --}}
	\rhead{\thetitle}
	\lfoot{}
	\rfoot{}
	\cfoot{}
	\pagestyle{fancy}
\pagestyle{empty}

\sffamily

%Kopfzeile

\author{Moritz Lenz}
\title{Spintronics}
\begin{document}
\maketitle

This is an attempt to reproduce what Khodas wrote in {\em Spin 
Polarization of Nonmagnetic Heterostructures: The Basics of Spin
Optics}, PRL 92.086602.

The setup consists of a 2D electron gas in the $x-z$ plane, where the
strength of the spin orbit interaction is a step function in $x$:
$\alpha(x) = \alpha \Theta(x)$. The region $x < 0$ is called the
"normal region", abbreviated with N, and the region with $x > 0$ is
called the "spin orbit" region, abbreviated as SO.

The Hamiltonian looks like this:

\begin{align}
    H_r &= \frac{p^2}{2m} + (-\vec y \times \vec \sigma) \cdot
            \alpha(x) \vec p\\ 
    p^2 &= p_x^2 + p_z^2
\end{align}

With the eigenvalues and the velocities

\begin{align}
    E_{\pm} &= \frac{p^2}{2m} \pm \alpha \\
    v_{\pm} &= \frac{\partial E_{\pm}}{\partial p} = \frac{p}{m} \pm \alpha
\end{align}

When a wave travels from the N to the SO region it's energy doesn't
change. Since its dispersion relation changes, the momentum must also
change. From here on when we write $p$ we mean the momentum in the N
region. The momentum in the SO region then follows as

\begin{align}
    \label{eq:pso}
    p_{SO}^{\pm} &= m v_F (\sqrt{1 + \tilde \alpha} \mp \tilde \alpha) \\
    \tilde\alpha &= \frac{\alpha}{v_F}
\end{align}

$p_z$ is conserved at the interface.

Solving the eigenvalue equation leads us to the eigenvectors in the SO
region:

\begin{align*}
   \chi_{SO}^{\pm} &= \frac{1}{n_{\pm}} 
                      \vect{p_{x,SO}^{\pm} \pm p_{SO}^\pm}{p_z} \\
    n_{\pm}^2      &= (p_{x,SO}^{\pm} \pm p_{SO}^\pm)^2 + p_z^2
\end{align*}

Where the lower index $x$ means that the value is projected onto the
$x$ axis. The angle between the $x$ axis and the momentum of the
incident wave is called $\phi$, so that $p_x = p \cos \phi$.

Note that in the N regime $H$ is a diagonal matrix, and the direction
of the eigenvectors can be chosen with some freedom. We pick
$\chi_N^{pm} = \lim_{\alpha \mapsto 0} \chi_{SO}^{\pm}$ to ensure that
$<\chi_N^+|\chi_{SO}^+> = 1$ holds true.


The overall wave function consists of an incident wave, 
and reflected and transmitted part:

\begin{align}
    \Psi^+ = e^{i p_z z} * \left\{
        \begin{array}{ll}
            e^{i p_x x} \chi_N^+ + e^{- i p_x x} (\chi_N^+ r_{++} +
                    \chi_N^- r_{-})     & x < 0\\
            e^{i p_x^+ x} \chi_{SO}^+ t_{++} + e^{-i p_x^- x}
            \chi_{SO}^- t_{-+}          & x > 0
        \end{array} \right.
\end{align}

The coefficient $r_{-+}$ is the amplitude with which the incident wave
of $+$ chirality is reflected into $-$ chirality etc.

To obtain the values for these coefficients one has to solve the
boundary conditions at the interface. The wave function is continuous
and the current is conserved, so $\frac{\partial H}{p_x} \Psi$ is also
continuous.

\begin{align}
    \Psi_N|_{x = -0}    &= \Psi_{SO}|_{x = +0}\\
    \left.\frac{\hat p_x}{m} \Psi_N\right|_{x = -0}
                        &= \left. \left(\frac{\hat p_x}{m} -\alpha \sigma_z\right)
                                    \Psi_{SO}\right|_{x = +0}
\end{align}

The second equation can be evaluated with $\hat p_x = -i \partial_x$
(assuming $\hbar = 1$, as done in the rest of the calculation) and
carrying out the derivation (and multiplied by $m$), yielding

\begin{align}
    p_x \chi_N^+ (1 - r_{++}) - p_x \chi_N^- r_{-+}
        =& p_x^+ \chi_{SO}^+ t_{++} + p_x^- \chi_{SO}^- t_{-+} \nonumber\\
         &   - m \alpha \left( p_x^+ \sigma_z \chi_{SO}^+ t_{++} + p_x^-
                     \sigma_z \chi_{SO}^- t_{-+} \right)
\end{align}

Dividing it by  $p_x = p \cos \phi$: 

\begin{align}
    \chi_N^+ (1 - r_{++}) - \chi_N^- r_{-+}
        =& \frac{p_x^+}{p_x} \chi_{SO}^+ t_{++} + \frac{p_x^-}{p_x} \chi_{SO}^- t_{-+} \nonumber\\
         &   - \frac{\ta}{\cos \phi} \left( \frac{p_x^+}{p_x} \sigma_z \chi_{SO}^+
                 t_{++} + \frac{p_x^-}{p_x} \sigma_z \chi_{SO}^- t_{-+} \right)
\end{align}

Multiplying both equations with $\chi_N^+$ and $\chi_N^-$ gives us
four scalar equations:

\begin{align} 
    \label{eq:a1}
    1 + r_{++}  &= <\chi_{SO}^+|\chi_N^+> t_{++} +
                    <\chi_{SO}^-|\chi_N^+> t_{-+}\\
    \label{eq:a2}
        r_{-+}  &= <\chi_{SO}^+|\chi_N^-> t_{++} + <\chi_{SO}^-|\chi_N^-> t_{-+}\\
    \label{eq:b1}
    \cos \phi  (1 + r_{++})
                &= \frac{p_{x,SO}}{p} \left(<\chi_{SO}^+|\chi_N^+> t_{++} 
                        + <\chi_{SO}^-|\chi_N^+> t_{-+} \right) \nonumber \\
                &  + \ta \left(<\chi_{SO}^+|\sigma_z|\chi_N^+> t_{++} 
                        + <\chi_{SO}^-|\sigma_z|\chi_N^+> t_{-+} \right)\\
 \label{eq:b2}
    \cos \phi \  r_{-+}
                &= \frac{p_{x,SO}}{p} \left(<\chi_{SO}^+|\chi_N^-> t_{++} 
                        + <\chi_{SO}^-|\chi_N^-> t_{-+} \right) \nonumber\\
                &  + \ta \left(<\chi_{SO}^+|\sigma_z|\chi_N^-> t_{++} 
                        + <\chi_{SO}^-|\sigma_z|\chi_N^-> t_{-+} \right)
\end{align}

(The scalar products of the spinors are indicated in bracket notation
for clarity, even though they don't imply in integration over any
variable).

$(\ref{eq:a1}) - \frac{1}{\cos \phi} (\ref{eq:b1})$
and
$(\ref{eq:a2}) - \frac{1}{\cos \phi} (\ref{eq:b2})$ form a linear,
homogeneous system of equations, which obviously has the trivial
solution $t_{++}, t_{-+} = 0$, which isn't a very physical solution.

Introducing

\begin{align}
    \beta = \frac{p_{x,SO}}{p \cos \phi}
\end{align}

we obtain for the determinant of that equation system

\begin{align}
    d = \left(<\chi_{N}^+|\chi_{SO}^+>( \beta - 1) +  
            \frac{\ta}{\cos \phi} <\chi_{N}^+|\sigma_z|\chi_{SO}^+>\right) \nonumber \\
        \cdot \left(<\chi_{N}^-|\chi_{SO}^->( \beta - 1) +  
            \frac{\ta}{\cos \phi} <\chi_{N}^-|\sigma_z|\chi_{SO}^->\right) \nonumber \\
     - \left(<\chi_{N}^-|\chi_{SO}^+>( \beta - 1) +  
            \frac{\ta}{\cos \phi} <\chi_{N}^+|\sigma_z|\chi_{SO}^->\right) \nonumber \\
        \cdot \left(<\chi_{N}^+|\chi_{SO}^->( \beta - 1) +  
            \frac{\ta}{\cos \phi} <\chi_{N}^+|\sigma_z|\chi_{SO}^->\right) \nonumber \\
\end{align}


Using the geometric relations

\begin{align}
    p^2         &= p_z^2 + p_x^2\\
    p_{SO}^2    &= p_z^2 + p_{x,SO}^2
\end{align}

together with eqn. \ref{eq:pso} we have all quantities to calculate
the scalar products.

As the Mathematica notebook \texttt{spinoptics-scratch.nb} shows,
a calculation up to the second order of $\ta$ gives us

\begin{align}
    d &= \ta^2 \sec^2 \phi \tan^2 \phi \nonumber\\
      &= \frac{\ta^2}{\cos^2 \phi}
\end{align}

So a not too approximative calculation gives no physical solutions to our boundary conditions.

\end{document}

% vim: ts=4 sw=4 expandtab spell spelllang=en_us tw=70

\NeedsTeXFormat{LaTeX2e}
\documentclass[11pt]{article}
%Absaetze nicht einruecken:
\usepackage{parskip}
\usepackage[utf8]{inputenc}
\usepackage[T1]{fontenc}
\usepackage{ae}
\usepackage[intlimits, sumlimits, namelimits]{amsmath}
\usepackage{bbm}
%Neue Macros fuer Mathe:
%in parentheses - gleich mit richtiger Groesse
\newcommand{\inp}[1]{\ensuremath{\left(#1\right)}}
\newcommand{\sqr}{\ensuremath{^{2}}}
\newcommand{\cube}{\ensuremath{^{3}}}
%Mengensymbole mit doppelten senkrechten Strichen:
\newcommand{\set}[1]{\ensuremath{\mathbbm{#1}}}
%definiert eine Norm, also zwei senkrechte Striche auf jeder Seite:
\newcommand{\norm}[1]{\ensuremath{\left|#1\right|}}
\newcommand{\vect}[2]{\ensuremath{\inp{\hspace{-.8ex}\begin{array}{c}#1\\#2\end{array}\hspace{-.4ex}}}}
%\newcommand{\vec3}[3]{\ensuremath{\inp{\hspace{-.8ex}\begin{array}{r}#1\\#2\\#3\end{array}\hspace{-.4ex}}}}
\newcommand{\entspr}{\ensuremath{\,\,\hat{=}\,\,}}%
\newcommand{\dx}[1][x]{\ensuremath{\textnormal d #1}}

% For stupid thinkos:
\newcommand{\cross}{\times}

% Tabellen:
\usepackage{array}
\setlength{\extrarowheight}{.2mm}
%Links im Text:
\usepackage{hyperref}
\usepackage{graphics,graphicx,fancyvrb}
%Raender einstellen
%\usepackage[a4paper, margin=15mm, top=30mm]{geometry}
\usepackage[a4paper]{geometry}
%Kopf - und Fusszeilen
\usepackage{lastpage}
\usepackage{fancyhdr}
	\lhead{Moritz Lenz}
	\chead{\bfseries{-- \thepage\ --}}
	\rhead{\thetitle}
	\lfoot{}
	\rfoot{}
	\cfoot{}
	\pagestyle{fancy}
\pagestyle{empty}

\sffamily

%Kopfzeile

\author{Moritz Lenz}
\title{Spintronics}
\begin{document}
\maketitle

This is an attempt to reproduce what Khodas wrote in {\em Spin 
Polarization of Nonmagnetic Heterostructures: The Basics of Spin
Optics}, PRL 92.086602.

The setup consists of a 2D electron gas in the $x-z$ plane, where the
strength of the spin orbit interaction is a step function in $x$:
$\alpha(x) = \alpha \Theta(x)$. The region $x < 0$ is called the
"normal region", abbreviated with N, and the region with $x > 0$ is
called the "spin orbit" region, abbreviated as SO.

The Hamiltonian looks like this:

\begin{align}
    H_r &= \frac{p^2}{2m} + (-\vec y \times \vec \sigma) \cdot
            \alpha(x) \vec p\\ 
    p^2 &= p_x^2 + p_z^2
\end{align}

With the eigenvalues and the velocities

\begin{align}
    E_{\pm} &= \frac{p^2}{2m} \pm \alpha \\
    v_{\pm} &= \frac{\partial E_{\pm}}{\partial p} = \frac{p}{m} \pm \alpha
\end{align}

When a wave travels from the N to the SO region it's energy doesn't
change. Since its dispersion relation changes, the momentum must also
change. From here on when we write $p$ we mean the momentum in the N
region. The momentum in the SO region then follows as

\begin{align*}
    p_{SO}^{\pm} &= m v_F (\sqrt{1 + \tilde \alpha} \mp \tilde \alpha)\\
    \tilde\alpha &= \frac{\alpha}{v_F}
\end{align*}

$p_z$ is conserved at the interface.

Solving the eigenvalue equation leads us to the eigenvectors in the SO
region:

\begin{align*}
   \chi_{SO}^{\pm} &= \frac{1}{n_{\pm}} 
                      \vect{p_{x,SO}^{\pm} \pm p_{SO}^\pm}{p_z} \\
    n_{\pm}^2      &= (p_{x,SO}^{\pm} \pm p_{SO}^\pm)^2 + p_z^2
\end{align*}

Note that in the N regime $H$ is a diagonal matrix, and the direction
of the eigenvectors can be chosen with some freedom. We pick
$\chi_N^{pm} = \lim_{\alpha \mapsto 0} \chi_{SO}^{\pm}$ to ensure that
$<\chi_N^+|\chi_{SO}^+> = 1$ holds true.

The overall wave function consists of an incident wave, and reflected
and transmitted part:

\begin{align}
    \Psi^+ = e^{i p_z z} * \left\{
        \begin{array}{ll}
            e^{i p_x x} \chi_N^+ + e^{- i p_x x} (\chi_N^+ r_{++} +
                    \chi_N^- r_{-})     & x < 0\\
            e^{i p_x^+ x} \chi_{SO}^+ t_{++} + e^{-i p_x^- x}
            \chi_{SO}^- t_{-+}          & x > 0
        \end{array} \right.
\end{align}

The coefficient $r_{-+}$ is the amplitude with which the incident wave
of $+$ chirality is reflected into $-$ chirality etc.

To obtain the values for these coefficients one has to solve the
boundary conditions at the interface

\begin{align}
    \Psi_N(x = -0)              &= \Psi_{SO}(x = +0)\\
    \frac{p_x}{m} \Psi_N(x = -0)&= \left(\frac{p_x}{m} -\alpha \sigma_z\right)
                                        \Psi_{SO}(x = +0)
\end{align}


\end{document}

% vim: ts=4 sw=4 expandtab spell spelllang=en_us tw=70

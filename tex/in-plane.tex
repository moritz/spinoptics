\NeedsTeXFormat{LaTeX2e}
\documentclass[11pt]{article}
%Absaetze nicht einruecken:
\usepackage{parskip}
\usepackage[utf8]{inputenc}
\usepackage[T1]{fontenc}
\usepackage{ae}
\usepackage[intlimits, sumlimits, namelimits]{amsmath}
\usepackage{bbm}
%Neue Macros fuer Mathe:
%in parentheses - gleich mit richtiger Groesse
\newcommand{\inp}[1]{\ensuremath{\left(#1\right)}}
\newcommand{\sqr}{\ensuremath{^{2}}}
\newcommand{\cube}{\ensuremath{^{3}}}
%Mengensymbole mit doppelten senkrechten Strichen:
\newcommand{\set}[1]{\ensuremath{\mathbbm{#1}}}
%definiert eine Norm, also zwei senkrechte Striche auf jeder Seite:
\newcommand{\norm}[1]{\ensuremath{\left|#1\right|}}
%Spaltenvektor - dreidimensional:
\newcommand{\svec}[3]{\ensuremath{\inp{\hspace{-.8ex}\begin{array}{r}#1\\#2\\#3\end{array}\hspace{-.4ex}}}}
\newcommand{\entspr}{\ensuremath{\,\,\hat{=}\,\,}}%
\newcommand{\dx}[1][x]{\ensuremath{\textnormal d #1}}
\newcommand{\dell}{\partial}

% For stupid thinkos:
\newcommand{\cross}{\times}

% Tabellen:
\usepackage{array}
\setlength{\extrarowheight}{.2mm}
%Links im Text:
%\usepackage{hyperref}
\usepackage{graphics,graphicx,fancyvrb}
%Raender einstellen
%\usepackage[a4paper, margin=15mm, top=30mm]{geometry}
\usepackage[a4paper]{geometry}
%Kopf - und Fusszeilen
\usepackage{lastpage}
\usepackage{fancyhdr}
	\lhead{Moritz Lenz}
	\chead{\bfseries{-- \thepage\ --}}
	\rhead{\thetitle}
	\lfoot{}
	\rfoot{}
	\cfoot{}
	\pagestyle{fancy}
\pagestyle{empty}

\sffamily

%Kopfzeile

\author{Moritz Lenz}
\title{In-plane magnetic field}
\begin{document}
\maketitle

We consider a perfect 2D system in the $x$-$y$-plane, to which we want to add
a magnetic field in $y$-direction, $\vec B = (0, B, 0)^T$. We can find 
a vector potential $\vec A = (B \cdot z, 0, 0)^T$ so that $\vec B = \nabla
\times \vec A$.

Now we want to discretize our Hamiltonian in a tight-binding model. We
approximate the derivation as $\dell_x F(x) = (F(x+a)-F(x-a))/(2a)$, and
obtain for the momentum operator:

\begin{equation}
    p_x \Psi(x) = \frac{\hbar}{i} \dell_x \Psi(x)=  \frac{\hbar}{i} \frac{\Psi(x+a) -
        \Psi(x-a)}{2a}
\end{equation}

Since $\dell_x = \frac{i}{\hbar} p_x$ is the generator of the translation, we can write

\begin{align}
    \Psi(x-a)   = e^{\dell_x a)} \Psi(x)  = e^{\frac{i}{\hbar} p_x} \Psi(x) 
\end{align}

And with the Peirls' substitution for introducing the magnetic field $\vec p
\rightarrow \vec p - q \vec A$:

\begin{equation} \label{eq:inplane}
    p_x \Psi(x) = \frac{\hbar}{i 2a} \inp{
          e^{\frac{-i}{\hbar} q A_x a} \Psi(x+a)
        + e^{\frac{ i}{\hbar} q A_x a} \Psi(x-a)
    } 
\end{equation}

(Since $[A_i, p_i] = 0$ for $i = x, y, z$ we don't run into any problems
 with this substitution).

Our earlier choice of gauge implies $A_y = 0$ and $A_x = B z$. $z$ can be set
to zero (corresponding to a translation in $z$ direction, which leaves our
system invariant), so the exponentials in (\ref{eq:inplane}) come out as $1$,
so there are no orbital contributions from an in-plane magnetic field.

From a physical point of view this is easy to understand: an electron (or
hole) moving parallel to the magnetic field isn't affected by the field, and
one moving in $x$-direction will experience a force in $z$-direction. But in
our model the electron is perfectly confined in the $x$-$y$ plane, so our
model can't handle that force.

Or putting it differently, the confinement into our plane corresponds to a
constraining force along the $z$ axis, which completely absorbs the force
resulting from a magnetic field.

\end{document}


% vim: ts=4 sw=4 expandtab spell spelllang=en_us



Numerical transport calculations have a huge advantage over the analytical
calculation: once a formalism is implemented, it works for arbitrary
configurations within the limits of the model. However it doesn't generally
supply us with data that enhances physical understanding, so its best use as a
means to expand on analytical calculations which we already understand.

In that spirit we performed numeric calculations, simulating a sample with
Rashba Spin-orbit coupling, and spin-resolved leads attached at the edges.

The calculation follows this rough scheme:

\begin{itemize}
    \item Set up the Hamiltonian $H$
    \item Calculate the self-energy matrices $\Sigma_p$
    \item Calculate the Green's functions $G^A$ and $G^R$ by inverting
          $H + \sum_p \Sigma_p$
    \item Use the Fisher-Lee relation to calculate the transmission matrix $T$
            from $G^R$, $G^A$ and $\Sigma_p$
\end{itemize}

\section*{Performance consideration}

These numeric calculations are computationally expensive. If we simulate a
lattice with a total of $n$ sites, the matrices $H$, $\Sigma_p$, $G^R$ and
$G^A$ are of dimensions $2n^2 \times 2n^2$
